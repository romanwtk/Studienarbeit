% !TEX root =  master.tex
\chapter{Zusammenfassung und Diskussion}
\section{Zusammenfassung}
In dieser Arbeit wurden kryptographische Verfahren, die auf fehlerkorrigierenden Codes wie \textit{Goppa}- oder verallgemeinerten \textit{Reed-Solomon}-Codes basieren, hinsichtlich ihres mathematischen Fundaments, ihrer Konfiguration und Bedeutung für die \textit{Post-Quanten-Kryptographie} dargestellt und diskutiert.
\paragraph{Fehlerkorrigierende Codes}
Fehlerkorrigierende Codes sind ein Teilgebiet der Codierungstheorie und der Kanalcodierung zuzuordnen, da ihr Zweck darin besteht, Informationen vor Übertragungsfehlern zu schützen, indem redundante Bits hinzugefügt werden. Je nach verwendetem Code kann so eine Anzahl an Fehlern \textit{erkannt} beziehungsweise \textit{korrigiert} werden. Die typischerweise zu diesem Zweck verwendeten linearen Codes basieren auf der Interpolation von Polynomen und werden zumeist auf endlichen Körpern der Form $\mathbb{F}_q$ definiert, wobei $q$ eine Primzahlpotenz ist. Sehr weit verbreitet, beispielsweise auch in \textit{\ac{QR}}-Codes, sind \textit{Reed-Solomon-Codes}. Kryptographisch bedeutsam ist die verallgemeinerte Form dieser Codes,  $GRS$-Codes, bei der die gewählten Auswertungsstellen des zu interpolierenden Polynoms durch ein Gewichtstupel augmentiert werden. \\\\Zu linearen Codes können \textit{duale} Codes gebildet werden. Für die Codierung verwendet werden nun \textit{Generator-} und \textit{Paritätsprüfmatrizen} eines Codes, wobei eine \textit{Generatormatrix} eines Codes die \textit{Paritätsprüfmatrix} des zu ihm dualen Codes ist. Die Kodierung eines Informationstupels kann \textit{systematisch} oder \textit{nicht-systematisch} erfolgen. Dafür wird eine Generatormatrix des Codes durch elementare Matrixoperationen so verändert, dass eine Submatrix der Generatormatrix eine Identitätsmatrix bildet. Die Dekodierung linearer Codes kann nun über \textit{Syndrome}, also Produkte aus empfangenem Wort und Paritätsprüfmatrix, erfolgen. In der Folge ensteht ein Gleichungssystem, über das fehlerhafte Stellen erkannt und korrigiert werden können, sofern die Fehlerkorrekturschranke des Codes eingehalten wurde. 
\paragraph{Kryptographische Verfahren}
Auf Basis dieser Codes wurden durch \textsc{McEliece} und \textsc{Niederreiter} jeweils \textit{Public-Key}-Kryptosysteme vorgeschlagen, wobei für Systeme diesen Typs charakteristisch ist, dass für Ver- und Entschlüsselung jeweils verschiedene Schlüssel verwendet werden und somit der Schlüssel zur Verschlüsselung veröffentlicht werden kann. In beiden Verfahren wird zunächst ein zugrundeliegender fehlerkorrigierender Code festgelegt. Dessen Generatormatrix im \textsc{McEliece}-Verfahren beziehungsweise dessen Paritätsprüfmatrix im \textsc{Niederreiter}-Schema wird nun permutiert und verwürfelt, um es Angreifenden unmöglich zu machen, die eigentliche Generatormatrix zu bestimmen. Die daraus entstandene Matrix stellt den öffentlichen Schlüssel dar. Die zur Permutation und Verwürfelung genutzten Matrizen fungieren nun als für die Entschlüsselung verwendeter privater Schlüssel und müssen geheim gehalten werden. Im \textsc{McEliece}-Verfahren wird die zu verschlüsselnde Nachricht nun mit dem öffentlichen Schlüssel multipliziert und zusätzlich wird ein zufälliger Störvektor hinzuaddiert. Im \textsc{Niederreiter}-Schema wird der öffentliche Schlüssel mit einem Klartextvektor verschlüsselt, der jedoch im Kontext der Dekodierung einem Fehlervektor entspricht. Die Entschlüsselung beider Verfahren verläuft nun durch die inverse Anwendung der Permutations- und Verwürfelungsmatrizen und der Anwendung eines Dekodieralgorithmus, wodurch der Klartext entsteht. Sofern beide Verfahren denselben Code verwenden, sind beide Verfahren äquivalent zueinander und haben identische Sicherheitseigenschaften.
\paragraph{Sicherheit und Bedeutung}
Die von \textsc{Niederreiter} für die Verwendung seines Kryptosystems vorgeschlagenen $GRS$-Codes sind für kryptographische Zwecke ungeeignet, da durch \textsc{Sidelnikov} und \textsc{Shestakov} ein Angriff publiziert wurde, der Struktureigenschaften dieser Codes ausnutzt und somit Fehlerstellen ohne Kenntnis des privaten Schlüssels wiederherstellen kann. Für die für das \textsc{McEliece}-Kryptosystem vorgesehenen \textit{Goppa}-Codes sind bei geeigneter Wahl der Parameter keine erfolgreichen Angriffe bekannt, weshalb das Verfahren als sicher gilt. Über entsprechende Parameter kann erreicht werden, dass das Verfahren sicher in Bezug auf \textit{adaptive chosen ciphertext}-Angriffe wird. Da beide Verfahren ferner auf mathematischen Problemen basieren, die als $\mathcal{NP}$-hart gelten, und Quantenalgorithmen wie \textit{Quantum Fourier Sampling} keine signifikanten Einschränkungen der Sicherheit erwarten lassen, gelten code-basierte Verfahren wie das \textsc{McEliece}-Kryptosystem als geeignete Kandidaten für die \textit{Post-Quanten-Kryptographie} und sind entsprechend Teil eines Standardisierungsprozesses des \ac{NIST}. In diesem Prozess sind auch Verfahren enthalten, die nicht auf \textit{Goppa}-Codes, sondern auf quasi-zyklischen \ac{MDPC}-Codes basieren und dadurch Vorteile in Bezug auf Rechenleistung und Schlüssellänge aufweisen.
\section{Reflexion}
Diese Arbeit stellt die theoretischen Hintergründe code-basierter Kryptographie dar und betrachtet insbesondere die Kryptosysteme von \textsc{Niederreiter} und \textsc{McEliece}. Relevante Algorithmen wie Kodierung, Dekodierung, Verschlüsselung und Entschlüsselung wurden zusätzlich zu einer Beschreibung in Text und mathematischen Ausdrücken im Mathematiksoftwaresystem \textit{SageMath} implementiert, um die einzelnen Algorithmen und ihre Korrektheit anschaulich werden zu lassen. Die Implementierungen sind jedoch nicht für andere als akademische Zwecke geeignet, da sie weder resistent gegenüber Seitenkanalangriffen sind, noch formal verifiziert wurden und zudem in der Standardausgabe eine Vielzahl von Informationen über die Algorithmen und ihre Parameter preisgegeben werden, was insbesondere für kryptographische Zwecke problematisch ist. \\\\
Der Fokus dieser Arbeit liegt weniger auf der Verifikation der Sicherheit und der Ausarbeitung möglicher Angriffe, sondern auf der algorithmischen Darstellung der Kryptosysteme und ihrer Hintergründe. In folgenden Arbeiten sollte daher ein stärkerer Fokus auf Versuche, die Sicherheit der Algorithmen und Kryptosysteme zu beweisen, gelegt werden und Angriffe wie jener von \textsc{Sidelnikov} und \textsc{Shestakov} implementiert werden. Weitere Verfahren der code-basierten Kryptographie, die vielleicht weniger populär, aber dennoch für eine tiefergehende Betrachtung geeignet sind, bedürfen ebenfalls weiterer Untersuchung, da sie hier lediglich in geringem Umfang Erwähnung finden.
\section{Ausblick}
Kryptographie, die auf fehlerkorrigierenden Codes basiert, erfährt durch das Forschungsfeld der \textit{Post-Quanten-Kryptographie} wesentlich mehr Aufmerksamkeit als zum Zeitpunkt der Veröffentlichung der entsprechenden Verfahren. Neben Verfahren, die auf \textit{multivariaten Polynomen} oder \textit{Gittern} basieren, ergibt sich durch code-basierte Verfahren ein in der Praxis neuartiger, wenn auch in der Theorie fundiert erforschter, Typus kryptographischer Verfahren, der für die Informationssicherheit der Zukunft mit hoher Wahrscheinlichkeit bedeutend sein wird. Insbesondere jene Verfahren, die auf quasi-zyklischen Codes basieren, erscheinen erfolgsversprechend, da sie die Probleme der Originalpublikationen wie die großen Schlüssellängen oder Performanznachteile adressieren. Offene Forschungsfragen bestehen unter anderem darin, sichere und praktikable Implementierungen der Systeme zu finden. Ebenso wird die Verifizierbarkeit der Sicherheit und jene einzelner Implementierungen einen Forschungsgegenstand darstellen. Es ist davon auszugehen, dass die Forschung in diesen Bereichen mit Abschluss des \ac{NIST}-Standardisierungsprozesses und der voranschreitenden Entwicklung von Quantencomputern beschleunigt wird, da dann die Notwendigkeit besteht, etablierte, jedoch nicht quantensichere, kryptographische Primitiven durch neue Verfahren der \textit{Post-Quanten-Kryptographie} zu ersetzen. Mit Quantencomputern wird mit hoher Wahrscheinlichkeit ebenfalls die Forschung zur Kryptoanalyse der neuen Verfahren verstärkt werden.