% !TEX root =  master.tex
\tikzstyle{block} = [rectangle, draw, fill=blue!20, text width=6em, text centered, minimum height=3em]
\chapter{Codierungstheorie}
Da im Rahmen dieser Arbeit ein kryptographisches Verfahren entwickelt wird, das auf Elementen der Codierungstheorie basiert, wird diese nun zunächst in Definitionen und Hintergründen motiviert.
\section{Grundbegriffe}
Sowohl in der Kryptographie, als auch in der Codierungstheorie fungieren \textbf{Nachrichten}, die über mit bestimmten Eigenschaften behaftete \textbf{Kanäle} übertragen werden, als die Subjekte der Anschauung.
\begin{definition}
Eine \textbf{Nachricht} $m$ sei definiert als eine endliche Folge von Zeichen $a_i \in \Sigma$, wobei $\Sigma$ eine endliche Menge von Zeichen (genannt \textbf{Alphabet}) bezeichnet.
\[m = (a_1, a_2, ..., a_{n-1}, a_n) \quad \forall i = 1, ..., n: a_i \in \Sigma\]
\end{definition}
Ein typisches Alphabet sind die Zeichen der ASCII-Kodierung, mit denen nahezu alle Worte und Sätze der natürlichen englischen Sprache gebildet werden können \parencite[vgl. ][]{rfc20}. Dieses Alphabet besteht nun nicht aus Zeichen der natürlichen Sprache, sondern aus 7-Bit-langen Zahlenwerten, was die Anwendung von Codes oder kryptographischen Verfahren ermöglicht. Im Rahmen dieser Arbeit wird implizit angenommen, dass Zeichen stets in einem Zahlenformat repräsentiert werden. \\\\
Die Definition einer informationstheoretischen Nachricht impliziert eine Autorenschaft, folglich muss jeder Nachricht eine Partei (ein natürliche Person, ein System oder ein Dienst) zugeordnet werden können, die im Folgenden als \textbf{Sender}\footnote{Da sich die Anwendung der modernen Kryptographie sehr überwiegend mit dem Austausch von verschlüsselten Nachrichten zwischen Systemen und nicht unmittelbar zwischen natürlichen Personen befasst, wird hier die männliche Form verwendet (Sender = Dienst/System).} der Nachricht bezeichnet wird. \\
Wird diese Nachricht nun über einen Kanal an eine andere Partei übertragen, so nennen wir diese den \textbf{Empfänger}. Entgegen der in der Kryptographie üblichen \textit{Alice-Bob}-Notation wird diese Terminologie beibehalten, um an den codierungstheoretischen Hintergrund anzuknüpfen. \\\\
Ein \textbf{Kanal} bezeichne ein Medium zur Datenübertragung wie beispielsweise einen elektrischen Leiter, einen Lichtwellenleiter oder die Luft für eine drahtlose Verbindung.
\begin{definition}
Ein \textbf{Kanal} sei definiert als ein Tupel $\langle e, d, g, s \rangle$, wobei $e$ das Medium, $d$ den Durchsatz, $g$ die Übertragungsgüte und $s$ den Vertraulichkeitsgrad des Kanals bezeichne. \\\\
Des Weiteren sei die Übertragungsfunktion $f: K \times M \mapsto M$ definiert als:
\[f(k, m) \to m'\] mit $m, m' \in M$ wobei $M$ die Menge aller Nachrichten ist und $k\in K$ wobei $K$ die Menge aller Kanäle ist. 
\end{definition}
Diese Definition impliziert, dass die Datenübertragung nicht zwingend fehlerfrei erfolgt und die Beziehung $m = m'$ daher nur im Idealfall gilt. Diese Feststellung liefert die Begründung für die Codierungstheorie.
\begin{figure}[h!]
\centering
\begin{tikzpicture}[node distance = 2cm]
\node[block] (Sender) {Sender};
\node[draw, circle, right = of Sender] (Kanal) {Kanal};
\node[block, right = of Kanal] (Empf) {Empfänger};
\draw[-] (Sender) -- node[midway, above] {$m$} (Kanal);
\draw[->] (Kanal) -- node[midway, above] {$m'$} (Empf);
\end{tikzpicture}
\caption{Gegenstand der Codierungstheorie (nach \parencite[][S. 1]{Willems2008})}
\end{figure}
\section{Problemstellung und Zielsetzung}

