% !TEX root =  master.tex
\chapter{Einleitung}
\begin{flushright}
{\glqq}The lesson here is that it is insufficient to protect ourselves with laws;
\\ we need to protect ourselves with mathematics.{\grqq} \\
-- \textsc{Bruce Schneier} in \parencite{Schneier2015}
\end{flushright}
In einer Welt, in der so viele Daten wie nie zuvor übertragen werden, digitale Kriegsführung und \textit{Nation-state attacks} nicht mehr bloß Gegenstand dystopischer Science-Fiction-Literatur, sondern Alltag sind, steigt die Relevanz kryptographischer Verfahren, die die Vertraulichkeit und Integrität schützenswerter Daten selbst unter der Annahme sicherstellen, dass Angreifenden nahezu unbegrenzte Ressourcen zur Verfügung stünden. \\\\
Das Forschungsgebiet der \textit{Post-Quanten-Kryptographie} \parencite[vgl. ][]{Bernstein2009} hat die Entwicklung kryptographischer Systeme zum Gegenstand, die selbst mit den durch Quantentechnologie anzunehmenden Rechenleistungssteigerungen nicht gebrochen werden können. Ein aussichtsreicher Kandidat dafür ist das \textsc{McEliece}-Kryptosystem, das auf linearen, fehlerkorrigierenden Codes basiert. Jene Verbindung der Codierungstheorie und Kryptographie ist Gegenstand dieser Arbeit: Ausgehend vom \textsc{McEliece}-Kryptosystem soll ein verwandter Ansatz von \textsc{Harald Niederreiter} betrachtet werden, der im Vergleich zum \textsc{McEliece}-Kryptosystem nicht auf \textit{Goppa}-, sondern auf \textit{Reed-Solomon}-Codes basiert und dadurch zwar bessere Rechenzeiten erreicht, jedoch auch an Sicherheit einbüßt. \\\\
Diese Arbeit stellt zunächst die theoretischen Hintergründe der Codierungstheorie für kryptographische Zwecke dar, bevor basierend darauf die Arbeiten von \textsc{McEliece} und \textsc{Niederreiter} analysiert werden. Abschließend wird die Sicherheit der Verfahren unter anderem anhand der Arbeit von \textsc{Sidelnikov} und \textsc{Shestakov} betrachtet und die Bedeutung jener Verfahren im Kontext der \textit{Post-Quanten-Kryptographie} diskutiert. \\\\
Ein Ziel dieser Arbeit ist es, die theoretischen Hintergründe hinter einzelnen Verfahren und Entwicklungen verständlich und dennoch fundiert zu vermitteln. Ergänzende Implementierungen in einem Mathematiksoftwaresystem sollen die Algorithmen verdeutlichen und zum Verständnis beitragen. Leser:innen dieser Arbeit sollen grundlegende Kenntnisse über code-basierte Kryptographie erlangen, um die aktuellen Entwicklungen nachvollziehen zu können.

\section{Thematische Übersicht}
\begin{figure}[h!]
\centering
\begin{tikzpicture}[mindmap, every node/.style={concept, execute at begin node=\hskip0pt}, concept color=blue!40, grow cyclic, level 1/.append style={level distance=4cm,sibling angle=90}, level 2/.append style={level distance=2.5cm,sibling angle=45}]
\node [root concept, concept color=orange!70] {Informations-theorie} % root
child { node[concept color=orange!70] {Kanal-codierung} 
	child { node[concept color=orange!70] {Blockcodes} 
		child { node[concept color=orange!70] {Lineare Codes} }
		child { node[concept color=orange!70] {BCH-Codes} }	
	}
	child { node {Faltungscodes} }
}
child { node[concept color=orange!70] {Kryptographie} 
	child { node {Blockchiffren} }
	child { node {Stromchiffren} }
	child { node[concept color=orange!70] {Code-based} 
		child { node[concept color=orange!70] {McEliece} }
		child { node[concept color=orange!70] {Knapsack} }	
	}
};
\end{tikzpicture}
\caption{Übersicht über für diese Arbeit {\color{orange}relevante} und {\color{blue}abgegrenzte} Teilgebiete der Informationstheorie}
\end{figure}
