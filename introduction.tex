% !TEX root =  master.tex
\chapter{Einleitung}
\begin{flushright}
{\glqq}The lesson here is that it is insufficient to protect ourselves with laws;
\\ we need to protect ourselves with mathematics.{\grqq} \\
-- \textsc{Bruce Schneier}
\end{flushright}
In einer Welt, in der so viele Daten wie nie zuvor übertragen werden, digitale Kriegsführung und \textit{Nation-state-attacks} nicht mehr nur Gegenstand dystopischer Science-Fiction-Literatur, sondern Alltag sind, steigt die Relevanz und die Systemkritikalität kryptographischer Verfahren, die es ermöglichen, die Vertraulichkeit und Integrität schützenswerter Daten selbst unter der Annahme, dass Angreifenden nahezu unbegrenzte Ressourcen zur Verfügung stehen sicherzustellen. \\\\
Das Forschungsgebiet der \textit{Post-Quanten-Kryptographie} hat die Entwicklung aussichtsreicher kryptographischer Systeme zum Gegenstand, die selbst mit den durch Quantentechnologie anzunehmenden Rechenleistungssteigerungen nicht gebrochen werden kann. Ein möglicher Kandidat dafür ist das \textit{McEliece}-Kryptosystem, das auf linearen, fehlerkorrigierenden Codes basiert. Jene Schnittmenge der Codierungstheorie und Kryptographie ist Gegenstand dieser Arbeit: Basierend auf dem McEliece-Kryptosystem soll ein aufbauender Ansatz von \textsc{Harald Niederreiter} betrachtet werden, das im Vergleich zum McEliece-Kryptosystem nicht auf \textit{Goppa}-, sondern auf \textit{Reed-Solomon}-Codes basiert. \\\\
Im Rahmen dieser Arbeit sollen zunächst die theoretischen Hintergründe der Codierungstheorie für kryptographische Zwecke dargelegt, bevor basierend darauf die Arbeiten von \textsc{McEliece} und \textsc{Niederreiter} analysiert und für die Entwicklung eines eigenen Kryptosystems genutzt werden. Auf jenes Verfahren werden abschließend Methoden der Kryptoanalyse nach \textsc{Sidelnikov} und \textsc{Shestakov} angewandt, um Aussagen über die Sicherheit des Verfahrens treffen zu können.
