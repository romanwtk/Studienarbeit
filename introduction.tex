% !TEX root =  master.tex
\chapter{Einleitung}
\begin{flushright}
{\glqq}The lesson here is that it is insufficient to protect ourselves with laws;
\\ we need to protect ourselves with mathematics.{\grqq} \\
-- \textsc{Bruce Schneier} in \parencite{Schneier2015}
\end{flushright}
In einer Welt, in der so viele Daten wie nie zuvor übertragen werden, digitale Kriegsführung und \textit{Nation-state-attacks} nicht mehr bloß Gegenstand dystopischer Science-Fiction-Literatur, sondern Alltag sind, steigt die Relevanz kryptographischer Verfahren, die es ermöglichen, die Vertraulichkeit und Integrität schützenswerter Daten selbst unter der Annahme, dass Angreifenden nahezu unbegrenzte Ressourcen zur Verfügung stehen, sicherzustellen. \\\\
Das Forschungsgebiet der \textit{Post-Quanten-Kryptographie} \parencite[vgl. ][]{Bernstein2009} hat die Entwicklung kryptographischer Systeme zum Gegenstand, die selbst mit den durch Quantentechnologie anzunehmenden Rechenleistungssteigerungen nicht gebrochen werden können. Ein aussichtsreicher Kandidat dafür ist das \textit{McEliece}-Kryptosystem, das auf linearen, fehlerkorrigierenden Codes basiert. Jene Verbindung der Codierungstheorie und Kryptographie ist Gegenstand dieser Arbeit: Basierend auf dem McEliece-Kryptosystem soll ein aufbauender Ansatz von \textsc{Harald Niederreiter} betrachtet werden, der im Vergleich zum McEliece-Kryptosystem nicht auf \textit{Goppa}-, sondern auf \textit{Reed-Solomon}-Codes basiert und dadurch zwar bessere Rechenzeiten erreicht, jedoch vermutlich auch an Sicherheit einbüßt. \\\\
Diese Arbeit stellt zunächst die theoretischen Hintergründe der Codierungstheorie für kryptographische Zwecke dar, bevor basierend darauf die Arbeiten von \textsc{McEliece} und \textsc{Niederreiter} analysiert und für die Entwicklung eines eigenen Kryptosystems genutzt werden. Auf jenes Verfahren werden abschließend Methoden der Kryptoanalyse unter Einbeziehung der Arbeit von \textsc{Sidelnikov} und \textsc{Shestakov} angewandt, um Aussagen über die Sicherheit des Verfahrens treffen zu können.
\pagebreak
\section{Thematische Übersicht}
\begin{figure}[h!]
\centering
\begin{tikzpicture}[mindmap, every node/.style={concept, execute at begin node=\hskip0pt}, concept color=blue!40, grow cyclic, level 1/.append style={level distance=4.5cm,sibling angle=90}, level 2/.append style={level distance=3cm,sibling angle=45}]
\node [root concept, concept color=orange!70] {Informations-theorie} % root
child { node[concept color=orange!70] {Kanal-codierung} 
	child { node[concept color=orange!70] {Blockcodes} 
		child { node[concept color=orange!70] {Lineare Codes} }
		child { node[concept color=orange!70] {BCH-Codes} }	
	}
	child { node {Faltungscodes} }
}
child { node[concept color=orange!70] {Kryptographie} 
	child { node {Blockchiffren} }
	child { node {Stromchiffren} }
	child { node[concept color=orange!70] {Code-based} 
		child { node[concept color=orange!70] {McEliece} }
		child { node[concept color=orange!70] {Knapsack} }	
	}
};
\end{tikzpicture}
\caption{Fortlaufend ergänzte Übersicht über relevante und abgegrenzte Teilgebiete der Informationstheorie}
\end{figure}
