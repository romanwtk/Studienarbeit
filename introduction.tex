% !TEX root =  master.tex
\chapter{Einleitung}
Während in verschiedenen Forschungsgruppen der Welt intensiv an Quantencomputern gearbeitet wird, versucht die Kryptographie bereits, Kryptosysteme zu entwickeln, die sich selbst mit hochperformanten Quantencomputern nicht brechen lassen. In dieser Hinsicht erweisen sich Kryptosysteme, die auf linearen fehlerkorrigierenden Codes basieren, als aussichtsreiche Kandidaten, da das Problem, aus einer codierten Nachricht mit einer Matrix entsprechend zusätzlich hinzugefügten Fehlern die Originalnachricht zu entschlüsseln, als hinreichend schwierig gilt. \\\\
Ansätze, kryptographische Verfahren auf Codierungstheorie zu basieren, wurden unter anderem von McEliece und Niederreiter vorgeschlagen. Während das abstrakte Konzept der Ver- und Entschlüsslung diesen Verfahren gemein ist, liegen die wesentlichen Unterschiede der Verfahren überwiegend in den zugrundeliegenden Codes. \\\\
Gegenstand dieser Arbeit ist das von Harald Niederreiter vorgeschlagene Schema eines Kryptosystems basierend auf Reed-Solomon-Codes. Auf dieser Arbeit aufbauend, wird ein Kryptosystem entwickelt, implementiert und analysiert. Ein weiterer Teil dieser Arbeit widmet sich anschließend möglichen Angriffen auf das vorgeschlagene Kryptosystem, wobei auf die Arbeit von Shostakov \& ... Bezug genommen wird. 