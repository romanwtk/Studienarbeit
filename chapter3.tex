% !TEX root =  master.tex
\chapter{Kryptographische Verfahren auf Basis fehlerkorrigierender Codes}
Dass fehlerkorrigierende Codes wie $GRS_k$-Codes für kryptographische Zwecke genutzt werden können, ergibt sich durch die Definition \textit{Code-basierter Kryptographie} von \textsc{\citeauthor{Sendrier2011}} wie folgt:
\begin{displayquote}
{\glqq}Code-based cryptography includes all cryptosystems, symetric or asymetric, whose security relies, partially or totally, on the hardness of decoding in a linear error correcting code, possibly chosen with some particular structure or in a specific family (for instance, quasi-cyclic codes, or Goppa codes).{\grqq} \parencite{Sendrier2011}
\end{displayquote}
Die Sicherheit dieser Verfahren liegt folglich darin begründet, dass es unter bestimmten Bedingungen bzw. bei der Wahl entsprechender Parameter, hinreichend schwierig ist, das Dekodierproblem linearer fehlerkorrigierender Codes zu lösen. \\\\
In diesem Kapitel werden die Ansätze dieser Verfahren zunächst allgemein dargestellt, bevor die Arbeit von \textsc{Niederreiter} genutzt wird, um ein Kryptosystem auf Basis von $GRS_k$-Codes zu konstruieren.
\section{Allgemeines}