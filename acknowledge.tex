% !TEX root =  master.tex
\chapter*{Danksagung}
Zuallererst möchte ich Ihnen, Herrn Professor Reinhold Hübl, ausdrücklich für Ihre herausragende und intensive Betreuung dieser Arbeit herzlich danken. Neben Ihrem stetigen konstruktiv-kritischen Blick auf dutzende Zwischenstände, der Suche nach ergänzenden Quellen oder sogar Ihrem Anfertigen zusätzlicher Ausarbeitungen sind es so zahlreiche zielführende und wegleitende Gedanken, die wesentlich dafür waren, dass diese Arbeit nun diese mich zufriedenstellende Gestalt annehmen konnte. Mein stetiger Wille, mich diesem mich herausfordernden Thema zu stellen und es mir nicht zu einfach zu machen, ist zu einem sehr großen Teil Ihrer so umfangreichen und wertschätzenden Unterstützung zu verdanken! \\\\
Auch fernab des inhaltlichen Austausches bin ich für Ihre Angebote und Ratschläge, die die Planung meines nächsten Studienabschnittes betreffen, außerordentlich dankbar -- auf Ihren Rat und Ihre Erfahrung zählen zu dürfen, ist eine wichtige Quelle meiner Zuversicht. \\\\
Außerdem möchte ich Ihnen, Herrn Professor Johannes Bauer, dafür danken, dass Sie mein Interesse an Kryptographie durch Ihre exzellente Vorlesung {\glqq}Kryptoanalyse und Methoden-Audit{\grqq} neu entfacht und die Brücke zwischen den mathematischen Hintergründen und der Implementierung kryptographischer Verfahren gebaut haben. Die -- stellenweise \textit{pfuschigen} -- \textit{SageMath}-Skripte im Anhang der Arbeit widme ich daher Ihnen, denn Sie stießen mich damals auf dieses geniale Werkzeug, dessen \textit{Fanbase} sich nun mit mir mindestens um eins inkrementiert hat. \\\\
Zu zweifeln und keine sonderlich hohe Meinung von den eigenen Fähigkeiten zu haben, ist ein Problem, mit dem ich vermutlich nicht alleine bin unter allen Informatiker:innen (vermutlich sogar Studierende und Forschende nahezu jeder Disziplin), die tagelang versuchen, den einen Fehler in der Logik zu finden, der perfiderweise konsequent die Lösungen verfälscht. Deshalb möchte ich dir, meiner lieben Laila, sehr dafür danken, dass du nicht aufhörst, an mich zu glauben und mich kontinuierlich mit deiner Liebe bestärkst. Auch für deine intensive Recherchehilfe, dein Weiterleiten von Quellen außerhalb meines Zugriffs und deine Erfahrung bezüglich des wissenschaftlichen Arbeitens, die ich mir nun langsam selbst erarbeiten möchte, danke ich dir sehr herzlich. Es tut so gut, dass es dich gibt! Ich freue mich auf unsere zukünftigen gemeinsamen Projekte.




