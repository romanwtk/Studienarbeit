% !TEX root =  master.tex
\chapter{Bedeutung in Gegenwart und Zukunft}
Ziel dieses Kapitels ist es, die Bedeutung der vorgestellten Verfahren nach \textsc{McEliece} und \textsc{Niederreiter} und auch der codebasierten Kryptographie im Allgemeinen anhand den Dimensionen Sicherheit, Performanz und Anwendbarkeit zu diskutieren, jeweils auch in Hinblick auf den erwarteten Zeitpunkt, an dem ein funktionsfähiger Quantencomputer in der Lage ist, herkömmliche kryptographische Verfahren zu brechen. Dass die hier aufgeführten Argumente lediglich für die Gegenwart Aussagekraft besitzen, liegt in der Natur der Sache, schließlich ist weder bekannt, \textit{wann} jener Quantencomputer existieren wird, noch \textit{mit welchen Möglichkeiten}.
\section{Sicherheit}
Da bereits gezeigt wurde, dass die Sicherheit der beiden Verfahren nach \textsc{Niederreiter} und \textsc{McEliece} die gleiche Sicherheitsklasse aufweisen, sofern dieselbe Klasse an Codes verwendet wird, liegt es nahe, die Sicherheit einzelner Codeklassen hinsichtlich ihrer Eignung für die Kryptographie zu beurteilen. Darüber hinaus werden exemplarisch weitere Sicherheitseigenschaften beider Verfahren dargestellt.
\subsection{Strukturangriffe auf $GRS$-Codes}
\textsc{Sidelnikov} und \textsc{Shestakov} zeigen in ihrem 1992 erschienenen Artikel \citetitle{Sidelnikov1992}, dass die insbesondere für das \textsc{Niederreiter}-Schema vorgesehenen $GRS$-Codes eine strukturelle Schwäche aufweisen: Sei $k_{pub} = H \cdot U$, wobei $U = S^{T} \cdot P$ und $H, S, P$ entsprechend der Definition des \textsc{Niederreiter}-Schemas eine Paritätsprüf-, Stör- und Permutationsmatrix darstellen. Die Autoren stellen nun in \parencite{Sidelnikov1992} einen Algorithmus vor, mit dem in einer polynomiellen Komplexität von $O((n-k-1)^4 + (n-k-1) \cdot n)$ die geheimen Matrizen $H$ und $U$ ermitteln werden können. Dies wird dadurch möglich, dass jeder Eintrag in einer Zeile $k_{{pub}_{i}}$ als Polynom mit einem Grad $\leq (n-k-1)$ aufgefasst werden kann. Es ergibt sich ein System polynomieller Gleichungen, dessen Lösung dem privaten Schlüssel $k_{priv} = \langle S, P\rangle$ entspricht \parencite[vgl. ][S. 120f.]{Overbeck2009}. Dieser Angriff begründet den Ausschluss von verallgemeinerten \textit{Reed-Solomon}-Codes aus der Liste geeigneter Codeklassen. Eine algorithmische Darstellung dieses Algorithmus ist in \parencite[][S. 18]{Engelbert2006} zu finden.
\subsection{\textit{Adaptive chosen ciphertext}-Angriffe}
\begin{definition}[\parencite{Kiltz2003}]
Sei $C$ ein \textbf{Public-Key-Kryptosystem} bestehend aus
\begin{itemize}
\item dem Schlüsselerzeugungsalgorithmus $K$,
\item dem Verschlüsselungsalgorithmus $E$ und
\item dem Entschlüsselungsalgorithmus $D$.
\end{itemize}
Ferner sei $A$ ein probabilistischer Angriff auf $C$, wobei $A$ aus zwei aufeinander folgenden Abläufen $A_1$ und $A_2$ mit folgenden Eigenschaften besteht:
\begin{enumerate}
\item $A_1$ erhält als Eingabe das öffentliche Ergebnis von $E$, also $k_{pub}$, und berechnet zwei Klartexte $m_0, m_1$ derselben Länge.
\item Nun wird für den Angriff verborgen $m_b$ mittels $k_{pub}$ verschlüsselt, wobei $b \in \lbrace 0, 1 \rbrace$ zufällig gewählt wird. Das entstandene Chiffrat wird als $c^*$ bezeichnet.
\item $A_2$ erhält dieses Chiffrat und versucht, $b$ zu ermitteln.
\end{enumerate}
Ein solcher Angriff $A$ wird als \textbf{\textit{Chosen plaintext}-Angriff (IND-CPA)} bezeichnet. Ein Kryptosystem ist dann \textbf{ununterscheidbar (IND)}, wenn der richtige Wert für $b$ mit einer Wahrscheinlichkeit von maximal $\frac{1}{2}$ erraten werden kann \parencite[vgl. ][S. 7]{bogdanov2005ind}.\\\\
Steht $A_1$ nur während des ersten Schritts ein Entschlüsselungsorakel zur Verfügung, das eine beliebige Anzahl an Chiffraten entschlüsselt, ohne dabei Informationen über den privaten Schlüssel preiszugeben, handelt es sich um einen \textbf{Nicht-adaptiven \textit{Chosen ciphertext}-Angriff (IND-CCA1)}. Steht zusätzlich auch in Schritt 3 ein solches Orakel zur Verfügung, das beliebige Chiffrate mit Ausnahme von $c^*$ entschlüsseln kann, so wird der Angriff $A$ als \textbf{Adaptiver \textit{Chosen ciphertext}-Angriff (IND-CCA2)} bezeichnet, da Angreifende durch die Orakel ihre Wahl der Klartexte $m_0$ und $m_1$ adaptieren können \parencite[vgl. ][S. 153f.]{Kiltz2003}.
\end{definition}
Die obig definierte IND-CCA2-Eigenschaft ist eine der stärksten bekannten Vorstellungen von Sicherheit von \ac{PKC}s \parencite[vgl. ][S. 6672]{Dottling2012} und fungiert damit natürlicherweise als Basis für die Parameterwahl. Unter anderem die Arbeiten von \textsc{\citeauthor{Dottling2012}} \parencite{Dottling2012} und \textsc{\citeauthor{Persichetti2018}} \parencite{Persichetti2018} zeigen, durch welche Parameter die Ununterscheidbarkeit von Codeworten selbst bei adaptiven \textit{chosen ciphertext}-Angriffen im \textsc{McEliece}-Kryptosystem erreicht werden kann. Die ursprünglich publizierte Form des \textsc{McEliece}-Kryptosystems ist nicht IND-CCA2-sicher \parencite[vgl. ][S. 34]{Bernstein2008}. \\\\
Insbesondere durch das Aufgeben von Restriktionen in der Wahl des zugrundeliegenden endlichen Körpers $\mathbb{F}_q$ und auch des Schlüsselerzeugungsverfahrens lässt sich die Sicherheit des \textsc{McEliece}-Kryptosystems signifikant zu der ursprünglich publizierten Variante erhöhen. So werden durch die Erzeugung des öffentlichen Schlüssels $k_{pub} = S \cdot G \cdot P$ beispielsweise Seitenkanalangriffe ermöglicht, weshalb es sogar sicherer sein kann, statt dieser Multiplikationen eine systematische Form $G'$ der Generatormatrix $G$ als Schlüssel zu veröffentlichen \parencite[vgl. ][S. 169]{Persichetti2018}, sofern durch Verwürfelung der Eingaben die Ununterscheidbarkeit gewährleistet wird \parencite[vgl. ][S. 34]{Bernstein2008}.

\subsection{\textit{Information Set Decoding}-Angriffe}
Eine weitere Klasse von Angriffen auf die Kryptosysteme von \textsc{Niederreiter} und \textsc{McEliece} sind \textit{Information Set Decoding}-Angriffe, die auf der bereits 1962 veröffentlichten Arbeit \citetitle{Prange1962} von \textsc{\citeauthor{Prange1962}} basieren und damit bereits \textsc{McEliece} bei der Wahl der Parameter des durch ihn vorgeschlagenen Kryptosystems beeinflusst haben. Veränderungen und Optimierungen dieses Angriffs, wie unter anderem beschrieben in \parencite{Bernstein2008} führen dazu, dass ausreichend große Parameter für $n$, $t$ und nach Möglichkeit auch endliche Körper $\mathbb{F}_q$ mit anderen Werten für $q$ als Zweierpotenzen gewählt werden müssen, um diesen Angriffen zu widerstehen  \parencite[vgl. ][S. 44ff.]{Bernstein2008} \parencite[vgl. ][S. 11]{SlidesLange}. \\\\
Diese Angriffe basieren darauf, über die Berechnung einer systematischen Form einer Generatormatrix eines Codes $C$ und einen gezielt erzeugten Vektor mit einer definierten Maximaldistanz zu $C$ zu erwirken, dass die Fehlerstellen einer Nachricht durch die systematische Form ersichtlich werden \parencite[vgl. ][]{Peters2010}. Diesen Angriffen kann durch die beschriebenen Maßnahmen begegnet werden \parencite[vgl. ][S. 44ff.]{Bernstein2008}.

\section{Anwendungen in der Gegenwart}
Während sich die Sicherheitsbewertung der Verfahren auch Dekaden nach ihrer Veröffentlichung nur geringfügig verändert hat und nie in einem Ausmaß, dem sich nicht durch Anpassungen der Parameter begegnen ließe, sind kaum Implementierungen und Anwendungen der Kryptosysteme von \textsc{McEliece} und \textsc{Niederreiter} bekannt \parencite[vgl. ][S. 44:2]{Maurich2015}. \\\\ 
Auch eine Webrecherche mit den Suchtermen {\glqq}Applications of McEliece{\grqq} und {\glqq}McEliece in practice{\grqq} am 05.04.2023 in den Suchmaschinen \textit{Google}, \textit{Google Scholar} und \textit{\href{https://eprint.iacr.org/}{Cryptology ePrint Archive (IACR)}} lieferte neben akademischen Beiträgen kaum Informationen über praktische Anwendungsszenarien und Implementierungen, die über die wissenschaftliche Betrachtung des Kryptosystems hinausgehen. Die Gründe dafür liegen womöglich in der in Relation zu anderen Verfahren großen Schlüssellänge von mehreren Kilo- bis Megabyte \parencite[vgl. ][S. 79]{Berger2009}. Einige Versuche, die Schlüssellängen des Verfahrens zu reduzieren, auch durch die Wahl anderer Codes, verringerten die Sicherheit des Verfahrens und erwiesen sich dadurch als ungeeignet \parencite[vgl. ][S. 44:2]{Maurich2015}. Andere Beiträge mit diesem Ziel wie \parencite{Berger2009} reduzierten die Schlüssellänge zwar, jedoch nicht auf ein mit anderen \ac{PKC}s vergleichbares Maß.

\section{Quantensicherheit}\label{PQC}
Dass code-basierte Kryptosysteme geeignet sein könnten, um die Schutzziele der Vertraulichkeit und Integrität auch unter der Verfügbarkeit von Quantencomputern zu gewährleisten, zeigen diverse Arbeiten, Forschungsgruppen und nicht zuletzt auch die Tatsache, dass entsprechende Kryptosysteme Teil von Standardisierungsrunden des \ac{NIST} sind. \\\\
In diesem Abschnitt wird begründet, warum das \textsc{McEliece}-Kryptosystem als Kandidat für \textit{Post-Quantum Cryptography} gilt. Anschließend werden drei Kryptosysteme aus dem Standardisierungsprozess des \ac{NIST} vorgestellt, die auf code-basierter Kryptographie und insbesondere auch auf den Kryptosystemen von \textsc{Niederreiter} und \textsc{McEliece} basieren.
\subsection{Argumentation}
Etablierte Verfahren wie \textit{RSA} oder \textit{ElGamal} basieren auf zahlentheoretischen Problemen wie dem diskreten Logarithmus oder der Primfaktorzerlegung \parencite{Dinh2011}. Während die Sicherheit auf konventionellen Computern durch die Abwesenheit effizienter Algorithmen zur Lösung dieser Probleme gewährleistet wird, zeigt sich durch Algorithmen wie jenem von \textsc{\citeauthor{Shor}} in \parencite{Shor}, der auf \textit{Quantum Fourier Sampling} basiert \parencite[vgl. ][S. 763]{Dinh2011} \parencite[vgl. ][S. 330]{QuantumFourier}, dass diese Verfahren mit Quantencomputern gebrochen werden können. \\\\\textsc{Dinh, Moore} und \textsc{Russell} zeigen in \parencite{Dinh2011}, dass \textit{Quantum Fourier Sampling}-Angriffe jedoch nicht die Sicherheit der Verfahren von \textsc{McEliece} und \textsc{Niederreiter} kompromittieren. Da diese Angriffe in einem Großteil der Quantenalgorithmen verwendet werden \parencite[vgl. ][S. 763]{Dinh2011}, entsteht die Einordnung dieser Kryptosysteme als potenziell quantensicher. Dass die beiden Kryptosysteme nicht resistent gegenüber anderen, möglicherweise noch nicht bekannten Quantenalgorithmen sind, bleibt jedoch möglich. Daher ist ein Ziel des \ac{NIST}-Standardisierungsprozesses, die Sicherheit der Kryptosysteme gründlich und langfristig zu erforschen \parencite[vgl. ][]{NISTPQC3}.
\subsection{Aktuelle Kandidaten}
Im \ac{NIST}-Standardisierungsprozess für Post-Quanten-Kryptographie werden in mehreren Runden Verfahren analysiert, hinsichtlich Design, Sicherheit und Performanz bewertet und bei entsprechender Eignung standardisiert \parencite[vgl. ][S. 27ff.]{NISTPQC3}. Eingereicht wurden und analysiert werden Verfahren, die auf fehlerkorrigierenden Codes, multivariaten Polynomen oder Gittern basieren. Ein ebenfalls eingereichtes Verfahren, das auf Isogenien aufgebaut ist, konnte total gebrochen werden \parencite[vgl. ][]{Decru} und ist daher nicht mehr Teil des Standardisierungsprozesses. Hingegen sind code-basierte Verfahren zwar noch nicht standardisiert, jedoch weiterhin Teil des Auswahlprozesses, zum Zeitpunkt der Abgabe dieser Arbeit in Runde vier. Im Folgenden werden die drei Verfahren \textit{BIKE}, \textit{Classic McEliece} und \textit{HQC} kurz beschrieben.
\subsubsection{BIKE}
\textit{BIKE} ist ein Akronym für \textit{Bit Flipping Key Encapsulation} und bezeichnet einen code-basierten \ac{KEM} auf Grundlage von quasi-zyklischen \ac{MDPC}-Codes \parencite[vgl. ][S. 29]{NISTPQC3}. Die Verschlüsselung während des \ac{PKE} folgt dabei dem \textsc{Niederreiter}-Schema.
\begin{definition}
Sei $C$ ein linearer $[n, k]$-Code. Falls eine Zahl $n_0 \in \mathbb{N}$ existiert, sodass für alle Codeworte $c \in C$ gilt, dass die Verschiebung von $c$ um $n_0$ ebenfalls ein gültiges Codewort aus $C$ ist, so ist $C$ \textbf{quasi-zyklisch} \parencite[vgl. ][S. 2070]{Tillich2013}.
\end{definition}
\begin{definition}
Ein $[n, k, d]$-\ac{MDPC}-Code ist ein linearer Code der Länge $n$, Dimension $k$, dessen  Paritätsprüfmatrizen in ihren Zeilen das konstante Gewicht $d$ aufweisen \parencite[vgl. ][S. 2070]{Tillich2013}.
\end{definition}
\textit{BIKE} sei seinen Autoren zufolge IND-CCA-sicher \parencite[vgl. ][S. 31]{NISTPQC3}. Die \ac{NIST} beabsichtigt, das Verfahren aufgrund seiner günstigen Performanzeigenschaften zum Ende der vierten Runde ihres Verfahrens zu standardisieren \parencite[vgl. ][S. 31]{NISTPQC3}.
\subsubsection{Classic McEliece}
Wie der Name andeutet, bezeichnet \textit{Classic McEliece} ein code-basiertes Kryptosystem, das in seinem Design nah an den klassischen Ansätzen von \textsc{Niederreiter} und \textsc{McEliece} bleibt. So wird ein binärer \textit{Goppa}-Code als Grundlage gewählt und die Verschlüsselung erfolgt wie im \textsc{Niederreiter}-Schema, wenn auch mit Modifikationen, um die Eigenschaft, IND-CCA-sicher zu sein, zu erreichen \parencite[vgl. ][S. 31f.]{NISTPQC3}. \\\\
Im Gegensatz zu \textit{BIKE} ist die Schlüssellänge und der daher benötigte Rechenaufwand nach Auffassung der \ac{NIST} zu hoch für eine Vielzahl von Anwendungen \parencite[vgl. ][S. 33]{NISTPQC3}. Dennoch wird die Einreichung in der vierten Runde weiter betrachtet, insbesondere, falls sich konkrete Anwendungsfälle ergeben, in denen konkurrierende Verfahren weniger gut geeignet scheinen als \textit{Classic McEliece} \parencite[vgl. ][S. 33]{NISTPQC3}.

\subsubsection{HQC (Hamming Quasi-Cyclic)}
Das Kryptosystem \textit{HQC} basiert wie \textit{BIKE} auf quasi-zyklischen \ac{MDPC}-Codes, deren Generatormatrizen jedoch in systematischer Form verwendet werden. Dafür ist die Definition zyklischer Matrizen erforderlich.
\begin{definition}
Sei $x = \langle x_1, \ldots, c_n\rangle \in \mathbb{F}_{q}^{n}$. Dann wird die aus $x$ abgeleitete Matrix
\begin{equation}
\text{rot}(x) = \left( 
\begin{array}{cccc}
x_1 & x_n & \cdots & x_2 \\
x_2 & x_1 & \cdots & x_3 \\
\vdots & \vdots & & \vdots \\
x_n & x_{n-1} & \cdots & x_1
\end{array}
\right)
\end{equation}
als \textbf{zyklisch} bezeichnet \parencite[vgl. ][S. 3929]{Aguilar-Melchor2018}.
\end{definition}
Mithilfe dieser zyklischen Matrizen können \textit{systematische quasi-zyklische} Codes angegeben werden \parencite[vgl. ][S. 3930f.]{Aguilar-Melchor2018}. \\\\
Das Verfahren weist nach Auffassung der \ac{NIST} eine akzeptable, wenn auch nicht optimale Performanz auf, wird aber insbesondere aufgrund der starken Sicherheitseigenschaften weiter im Standardisierungsprozess betrachtet \parencite[vgl. ][S. 34]{NISTPQC3}.