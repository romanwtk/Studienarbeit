% !TEX root =  master.tex
\chapter{Bedeutung in Gegenwart und Zukunft}
Ziel dieses Kapitels ist es, die Bedeutung der vorgestellten Verfahren nach \textsc{McEliece} und \textsc{Niederreiter} und auch der codebasierten Kryptographie im Allgemeinen anhand den Dimensionen Sicherheit, Performanz und Anwendbarkeit zu diskutieren, jeweils auch in Hinblick auf den erwarteten Zeitpunkt, an dem ein funktionsfähiger Quantencomputer in der Lage ist, herkömmliche kryptographische Verfahren zu brechen. Dass die hier aufgeführten Argumente lediglich für die Gegenwart Aussagekraft besitzen, liegt in der Natur der Sache, schließlich ist weder bekannt, \textit{wann} jener Quantencomputer existiert, noch \textit{mit welchen Möglichkeiten}.
\section{Sicherheit}
Da bereits gezeigt wurde, dass die Sicherheit der beiden Verfahren nach \textsc{Niederreiter} und \textsc{McEliece} die gleiche Sicherheitsklasse aufweisen, sofern dieselbe Klasse an Codes verwendet wird, liegt es nahe, die Sicherheit einzelner Codeklassen hinsichtlich ihrer Eignung für die Kryptographie zu beurteilen.
\subsection{Strukturangriffe auf $GRS$-Codes}
\textsc{\citeauthor{Sidelnikov1992}} zeigen in ihrem 1992 erschienenen Artikel \citetitle{Sidelnikov1992}, dass die insbesondere für das \textsc{Niederreiter}-Schema vorgesehenen $GRS$-Codes eine strukturelle Schwäche aufweisen: Sei $k_{pub} = H \cdot U$, wobei $U = S^{T} \cdot P$ und $H, S, P$ entsprechend der Definition des \textsc{Niederreiter}-Schemas eine Paritätsprüf-, Stör- und Permutationsmatrix darstellen. Die Autoren stellen nun in \parencite{Sidelnikov1992} einen Algorithmus vor, mit dem in einer polynomiellen Komplexität von $O((n-k-1)^4 + (n-k-1) \cdot n)$ die geheimen Matrizen $H$ und $U$ ermitteln werden können. Dies wird dadurch möglich, dass jeder Eintrag in einer Zeile $k_{{pub}_{i}}$ als Polynom mit einem Grad $\leq (n-k-1)$ aufgefasst werden kann. Es ergibt sich ein System polynomieller Gleichungen, dessen Lösung dem privaten Schlüssel $k_{priv} = \langle S, P\rangle$ entspricht \parencite[vgl. ][S. 120f.]{Overbeck2009}. Dieser Angriff begründet den Ausschluss von verallgemeinerten \textit{Reed-Solomon}-Codes aus der Liste geeigneter Codeklassen. Eine algorithmische Darstellung dieses Algorithmus ist in \parencite[][S. 18]{Engelbert2006}.
\subsection{\textit{Adaptive chosen ciphertext}-Angriffe}
\begin{definition}[\parencite{Kiltz2003}]
Sei $C$ ein \textbf{Public-Key-Kryptosystem} bestehend aus
\begin{itemize}
\item dem Schlüsselerzeugungsalgorithmus $K$,
\item dem Verschlüsselungsalgorithmus $E$ und
\item dem Entschlüsselungsalgorithmus $D$.
\end{itemize}
Ferner sei $A$ ein probabilistischer Angriff auf $C$, wobei $A$ aus zwei aufeinander folgenden Abläufen $A_1$ und $A_2$ mit folgenden Eigenschaften besteht:
\begin{enumerate}
\item $A_1$ erhält als Eingabe das öffentliche Ergebnis von $E$, also $k_{pub}$, und berechnet zwei Klartexte $m_0, m_1$ derselben Länge.
\item Nun wird für den Angriff verborgen $m_b$ mittels $k_{pub}$ verschlüsselt, wobei $b \in \lbrace 0, 1 \rbrace$ zufällig gewählt wird. Das entstandene Chiffrat wird als $c^*$ bezeichnet.
\item $A_2$ erhält dieses Chiffrat und versucht, $b$ zu ermitteln.
\end{enumerate}
Ein solcher Angriff $A$ wird als \textbf{\textit{Chosen plaintext}-Angriff (IND-CPA)} bezeichnet. Ein Kryptosystem ist dann \textbf{ununterscheidbar (IND)}, wenn der richtige Wert für $b$ mit einer von Wahrscheinlichkeit von maximal $\frac{1}{2}$ erraten werden kann \parencite[vgl. ][S. 7]{bogdanov2005ind}.\\\\
Steht $A_1$ nur während des ersten Schritts ein Entschlüsselungsorakel, das eine beliebige Anzahl an Chiffraten entschlüsselt, ohne dabei Informationen über den privaten Schlüssel preiszugeben, zur Verfügung, handelt es sich um einen \textbf{Nicht-adaptiven \textit{Chosen ciphertext}-Angriff (IND-CCA1)}. Steht zusätzlich auch in Schritt 3 ein solches Orakel zur Verfügung, dass beliebige Chiffrate mit Ausnahme von $c^*$ entschlüsseln kann, so wird der Angriff $A$ als \textbf{Adaptiver \textit{Chosen ciphertext}-Angriff (IND-CCA2)} bezeichnet, da Angreifende durch die Orakel ihre Wahl der Klartexte $m_0$ und $m_1$ adaptieren können \parencite[vgl. ][S. 153f.]{Kiltz2003}.
\end{definition}