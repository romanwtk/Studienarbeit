% !TEX root =  master.tex
\chapter{Bedeutung in Gegenwart und Zukunft}
Ziel dieses Kapitels ist es, die Bedeutung der vorgestellten Verfahren nach \textsc{McEliece} und \textsc{Niederreiter} und auch der codebasierten Kryptographie im Allgemeinen anhand den Dimensionen Sicherheit, Performanz und Anwendbarkeit zu diskutieren, jeweils auch in Hinblick auf den erwarteten Zeitpunkt, an dem ein funktionsfähiger Quantencomputer in der Lage ist, herkömmliche kryptographische Verfahren zu brechen. Dass die hier aufgeführten Argumente lediglich für die Gegenwart Aussagekraft besitzen, liegt in der Natur der Sache, schließlich ist weder bekannt, \textit{wann} jener Quantencomputer existiert, noch \textit{mit welchen Möglichkeiten}.
\section{Sicherheit}
Da bereits gezeigt wurde, dass die Sicherheit der beiden Verfahren nach \textsc{Niederreiter} und \textsc{McEliece} die gleiche Sicherheitsklasse aufweisen, sofern dieselbe Klasse an Codes verwendet wird, liegt es nahe, die Sicherheit einzelner Codeklassen hinsichtlich ihrer Eignung für die Kryptographie zu beurteilen.
\subsection{Strukturangriffe auf $GRS$-Codes}
\textsc{\citeauthor{Sidelnikov1992}} zeigen in ihrem 1992 erschienenen Artikel \citetitle{Sidelnikov1992}, dass die insbesondere für das \textsc{Niederreiter}-Schema vorgesehenen $GRS$-Codes eine strukturelle Schwäche aufweisen: Sei $k_{pub} = H \cdot U$, wobei $U = S^{T} \cdot P$ und $H, S, P$ entsprechend der Definition des \textsc{Niederreiter}-Schemas eine Paritätsprüf-, Stör- und Permutationsmatrix darstellen. Die Autoren stellen nun in \parencite{Sidelnikov1992} einen Algorithmus vor, mit dem in einer Komplexität von $O((n-k-1)^4 + (n-k-1) \cdot n)$ die geheimen Matrizen $H$ und $U$ ermitteln werden können. Dies wird dadurch möglich, dass jeder Eintrag in einer Zeile $k_{{pub}_{i}}$ als Polynom mit einem Grad $\leq (n-k-1)$ aufgefasst werden kann. Es ergibt sich ein System polynomieller Gleichungen, dessen Lösung dem privaten Schlüssel $k_{priv} = \langle S, P\rangle$ entspricht \parencite[vgl. ][S. 120f.]{Overbeck2009}.