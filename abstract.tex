% !TEX root =  master.tex
\chapter*{Kurzfassung (Abstract)}
\addcontentsline{toc}{chapter}{Kurzfassung (Abstract)}
\section*{Abstract}
While researchers in the field of cryptography get worried about the upcoming development of quantum computers, some well-studied but nevertheless not widely applied cryptosystems like the \textsc{McEliece} and \textsc{Niederreiter} scheme seem to be quantum resistent since they are not based on typical mathematical problems like the factorization or discrete logarithm problem. This thesis describes the theory behind those schemes including finite fields, polynomials and error-correcting codes like \textit{Reed-Solomon} or \textit{Goppa} codes. Furthermore two cryptosystems built on this class of codes are constructed, implemented in \textit{SageMath} and reviewed in terms of security and applicability. Both cryptosystems share the same security properties and are relevant for the \textit{Post Quantum Cryptography} research field, as some submissions to the NIST standardization process convey.
\section*{Zusammenfassung}
Durch die voranschreitende Entwicklung von Quantencomputern sieht sich das Forschungsfeld der Kryptographie vor neuen Herausforderungen. Umfangreich erforschte, aber in der Praxis kaum eingesetzte Kryptosysteme wie jene von \textsc{McEliece} und \textsc{Niederreiter} sind vielversprechende Kandidaten für quanten-resistente Verfahren, da sie auf anderen Problemen als dem Faktorisierungsproblem oder dem Problem des diskreten Logarithmus basieren. In dieser Arbeit werden zunächst die theoretischen Hintergründe anhand von endlichen Körpern, Polynomen und fehler-korrigierenden Codes wie \textit{Reed-Solomon}- oder \textit{Goppa}-Codes erläutert, bevor die beiden genannten Kryptosysteme definiert, im Mathematik-Softwaresystem \textit{SageMath} implementiert und in Hinblick auf Sicherheit und Anwendbarkeit verglichen werden. Beide Verfahren zeigen ähnliche Sicherheitseigenschaften und sind somit relevant für das Gebiet der \textit{Post-Quanten-Kryptographie}, da einige Einreichungen im NIST-Standardisierungsprozess auf diesen Verfahren basieren.
