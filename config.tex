% !TEX root =  master.tex

%%%%%%%%%%%%%%%%%%%%%%%%%%%%%%%%%%%%%%%%%%%%%%%%%%%%%%%%%%%%%%%%%%
%	ANLEITUNG: 
% Passen Sie gegebenenfalls alle Stellen im Dokument an, die mit 
% @stud 
% markiert sind.
%%%%%%%%%%%%%%%%%%%%%%%%%%%%%%%%%%%%%%%%%%%%%%%%%%%%%%%%%%%%%%%%%%

\usepackage{makeidx}                  % allows index generation
\usepackage{listings}	                %Format Listings properly
\usepackage{lipsum}                   % Blindtext
\usepackage{graphicx}                 % use various graphics formats
\usepackage[german]{varioref}         % nicer references \vref
\usepackage{caption}	                % better Captions
\usepackage{booktabs}                 % nicer Tabs
\usepackage{hyperref} % keine roten Markierungen bei Links
\usepackage{fnpct}                    % Correct superscripts 
\usepackage{calc}                     % Used for extra space below footsepline, in particular
\usepackage{array}
\usepackage{acronym}
\usepackage{algorithm}
\usepackage{algpseudocode}
\usepackage{setspace}
\usepackage{tocloft}
\usepackage{amsthm}
\usepackage{amssymb}
\usepackage{amsmath}
%\usepackage{ntheorem}
\usepackage{tikz}
\usetikzlibrary{positioning}
\usetikzlibrary{backgrounds}
\usetikzlibrary{shapes.geometric}
\usepackage{float}
\usetikzlibrary{positioning}
\usetikzlibrary{shapes.geometric}
\usetikzlibrary{mindmap}

%% Schriftarten- und Zeichenpakete
\usepackage[T1]{fontenc}
\usepackage[utf8]{inputenc}

%% THEOREME
\newtheoremstyle{break}% name
  {}%         Space above, empty = `usual value'
  {}%         Space below
  {}% Body font
  {}%         Indent amount (empty = no indent, \parindent = para indent)
  {\bfseries}% Thm head font
  {}%        Punctuation after thm head
  {\newline}% Space after thm head: \newline = linebreak
  {}% 

\theoremstyle{break}
\newtheorem{definition}{Definition}
\newtheorem{note}{Bemerkung}
\newtheorem{proposition}{Satz}
\newtheorem{theorem}{Theorem}
\newtheorem{example}{Beispiel}
\newtheorem{lemma}{Lemma}

\definecolor{mygreen}{rgb}{0,0.6,0}
\definecolor{mygray}{rgb}{0.5,0.5,0.5}
\definecolor{mymauve}{rgb}{0.58,0,0.82}

\lstset{ %
  backgroundcolor=\color{white},   % choose the background color
  basicstyle=\footnotesize,        % size of fonts used for the code
  breaklines=true,                 % automatic line breaking only at whitespace
  captionpos=b,                    % sets the caption-position to bottom
  commentstyle=\color{mygreen},    % comment style
  escapeinside={\%*}{*)},          % if you want to add LaTeX within your code
  keywordstyle=\color{blue},       % keyword style
  stringstyle=\color{mymauve},     % string literal style
}


%%
%% @stud
%%
%%	FONT SELECTION: Schriftarten und Schriftfamilie
%%%%%%%%%%%%%
%% SCHRIFTART
%%%%%%%%%%%%%
% 0) without decomment: normal font families 
% ...
% 1) Latin Modern 
%\usepackage{lmodern}        
% 2) Times 
%\usepackage{mathptmx}         
% 3) Helvetica
%\usepackage[scaled=.92]{helvet} 
%%%%%%%%%%%%%%%%%%
%%	SCHRIFTFAMILIE
%%%%%%%%%%%%%%%%%%
% ohne Serifen
\renewcommand*{\familydefault}{\sfdefault}
\addtokomafont{disposition}{\sffamily}
%
% mit Serifen
%\renewcommand*{\familydefault}{\rmdefault}
%\addtokomafont{disposition}{\rmfamily}
%
% Typewriter
%\renewcommand*{\familydefault}{\ttdefault}
%\addtokomafont{disposition}{\ttfamily}

%%
%% @stud
%%
%% LANGUAGE SETTINGS
\usepackage[ngerman]{babel} 	        % german language
\usepackage[german=quotes]{csquotes} 	% correct quoting using \enquote{}
%\usepackage[english]{babel}          % english language
%\usepackage{csquotes} 	              % correct quoting using \enquote{}

%%
%% @stud
%%
%% Uncomment the following lines to support hard URL breaks in bibliography 
%\apptocmd{\UrlBreaks}{\do\f\do\m}{}{}
%\setcounter{biburllcpenalty}{9000}% Kleinbuchstaben
%\setcounter{biburlucpenalty}{9000}% Großbuchstaben

%%
%% @stud
%%
%% FOOTNOTES: Count footnotes over chapters
%% \counterwithout{footnote}{chapter}

%	ACRONYMS
\makeatletter
\@ifpackagelater{acronym}{2015/03/20}
{\renewcommand*{\aclabelfont}[1]{\textbf{{\acsfont{#1}}}}}{}
\makeatother

%	LISTINGS
% @stud: ggf. Namen/Text anpassen (englisch)
\renewcommand{\lstlistingname}{Quelltext} 
\renewcommand{\lstlistlistingname}{Quelltextverzeichnis}
\lstset{numbers=left,
	numberstyle=\tiny,
	captionpos=b,
	basicstyle=\ttfamily\small}

%	ALGORITHMS
% @stud: ggf. Namen/Text anpassen (englisch)
\renewcommand{\listalgorithmname}{Algorithmenverzeichnis}
\floatname{algorithm}{Algorithmus}

%	PAGE HEADER / FOOTER
%	Warning: There are some redefinitions throughout the master.tex-file!  DON'T CHANGE THESE REDEFINITIONS!
\RequirePackage[automark]{scrlayer-scrpage}
%alternatively with separation lines: \RequirePackage[automark,headsepline,footsepline]{scrlayer-scrpage}

\renewcommand{\chaptermarkformat}{}
\RedeclareSectionCommand[beforeskip=0pt]{chapter}
\clearscrheadfoot

%\ifoot[\rule{0pt}{\ht\strutbox+\dp\strutbox}DHBW Mannheim]{\rule{0pt}{\ht\strutbox+\dp\strutbox}DHBW Mannheim}
\ofoot[\rule{0pt}{\ht\strutbox+\dp\strutbox}\pagemark]{\rule{0pt}{\ht\strutbox+\dp\strutbox}\pagemark}
\ohead{\headmark}

\newcommand{\TitelDerArbeit}[1]{\def\DerTitelDerArbeit{#1}\hypersetup{pdftitle={#1}}}
\newcommand{\AutorDerArbeit}[1]{\def\DerAutorDerArbeit{#1}\hypersetup{pdfauthor={#1}}}
%\newcommand{\Firma}[1]{\def\DerNameDerFirma{#1}}
\newcommand{\Kurs}[1]{\def\DieKursbezeichnung{#1}}
\newcommand{\Abteilung}[1]{\def\DerNameDerAbteilung{#1}}
\newcommand{\Studiengangsleiter}[1]{\def\DerStudiengangsleiter{#1}}
\newcommand{\WissBetreuer}[1]{\def\DerWissBetreuer{#1}}
%\newcommand{\FirmenBetreuer}[1]{\def\DerFirmenBetreuer{#1}}
\newcommand{\Bearbeitungszeitraum}[1]{\def\DerBearbeitungszeitraum{#1}}
\newcommand{\Abgabedatum}[1]{\def\DasAbgabedatum{#1}}
\newcommand{\Matrikelnummer}[1]{\def\DieMatrikelnummer{#1}}
\newcommand{\Studienrichtung}[1]{\def\DieStudienrichtung{#1}}
\newcommand{\ArtDerArbeit}[1]{\def\DieArtDerArbeit{#1}}
\newcommand{\Literaturverzeichnis}{Literaturverzeichnis}

\newcommand{\settingBibFootnoteCite}{
	\setlength{\bibparsep}{\parskip}		  % Add some space between biblatex entries in the bibliography
	\addbibresource{bibliography.bib}	    % Add file bibliography.bib as biblatex resource
	\DefineBibliographyStrings{ngerman}{andothers = {{et\,al\adddot}},}
}

\newcommand{\setTitlepage}{
	\input{titlepage}
	\pagenumbering{roman} % Römische Seitennummerierung
	\normalfont	
}

\newcommand{\initializeText}{
	\clearpage
	\ihead{\chaptername~\thechapter} % Neue Header-Definition
	\pagenumbering{arabic}           % Arabische Seitenzahlen
}

\newcommand{\initializeBibliography}{
	\ihead{}
	\printbibliography[title=\Literaturverzeichnis] 
	\cleardoublepage
}

\newcommand{\initializeAppendix}{
	\appendix
  \ihead{}
  \cftaddtitleline{toc}{chapter}{Anhang}{}
}

